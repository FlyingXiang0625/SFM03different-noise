\documentclass[12pt]{beamer}

\mode<presentation> {
	
	
	%\usetheme{Madrid}
	%\usepackage{times}
	\renewcommand{\familydefault}{\rmdefault}
	\usetheme{CambridgeUS}
	\usepackage[latin1]{inputenc}
	\usefonttheme{professionalfonts}
	\usepackage{times}
	\usepackage{tikz}
	\usepackage{amsmath}
	\usepackage{verbatim}
	\usepackage{enumerate}
	\usepackage{setspace}
	\usetikzlibrary{arrows,shapes}
	\usepackage{amsmath}
	\usepackage{eurosym}
	\usepackage{framed}
	\usepackage{extarrows}
}

\usepackage{graphicx}
\usepackage{booktabs}
\usepackage{url}

%----------------------------------------------------------------------------------------
%	TITLE PAGE
%----------------------------------------------------------------------------------------



\title{Different noises}

\author[Group ID: No.3]{Instructor:  Prof. H\"{a}rdle Group ID: No.3} % Your name
\institute[]{
	\textsl{Project ID: No.18}
}
\date[July 20$^{th}, 2016$]{} % Date, can be changed to a custom date







\begin{document}
	
\begin{frame}
	\titlepage
\end{frame}
\begin{frame}
	\frametitle{Outline}
	\tableofcontents
\end{frame}


\section{Introduction}

\begin{frame}
	\frametitle{Introduction}
	\quad In audio engineering, electronics,physics and many other fields, the color of a noise(a signal produced by a stochastic process) is generally considered to be some broad characteristic of its power spectrum. Different colors of noise have significantly different properties.
    
    \quad The practice of naming kinds of noise after colors started with white noise, a signal whose spectrum has
    equal power within any equal interval of frequencies. That name was given by analogy with white light, which was assumed to have such a flat power spectrum over the visible range. Other color names like red, pink and blue, were then given to noise with other spectral profiles, often (but not always) in reference to the color of light with similar spectra. Some of those names have standard definitions in certain disciplines, while others are very informal and poorly defined.


\end{frame}

\section{White noise}
\begin{frame}
	\frametitle{White noise}
	\quad White noise is a signal or process, which is named analogy to white light with a flat frequency spectrum when plotted as a linear function of frequency (e.g. in Hz). In other words, the signal has equal power in any band of a given bandwidth (power spectral density) when the bandwidth is measured in Hz.

\end{frame}


\section{Pink noise}

\begin{frame}
	\frametitle{Pink noise}
	\quad The frequency spectrum of pink noise is linear in logarithmic space; it has equal power in bands that are proportionally wide. This means that pink noise would have equal power in the frequency range from 40 to 60 Hz as in the band from 4000 to 6000 Hz.

\end{frame}

\section{Blue noise}
\begin{frame}
	\frametitle{Blue niose}
	\quad Blue noise is also called azure noise. Blue noise's power density increases 3 dB per octave with increasing frequency (density proportional to f) over a finite frequency range. In computer graphics, the term  blue noise is sometimes used more loosely as any noise with minimal low frequency components and no concentrated spikes in energy. This can be good noise for dithering. Retinal cells are arranged in a blue-noise-like pattern which yields good visual resolution.


\end{frame}

\section{Figure}
	\begin{frame}
		\frametitle{Figure}
\begin{figure}
  \centering
  % Requires \usepackage{graphicx}
  \includegraphics[width=200pt]{different-noise.jpg}\\\
 
\end{figure}
		
	\end{frame}
	
	%----------------------------------------------------------------------------------------
	
\end{document}
